\documentclass[10pt,a4paper]{article}

\usepackage[utf8]{inputenc}
\usepackage[dutch]{babel}
\usepackage{fancyhdr}
\usepackage{geometry}
\usepackage{graphicx}
\usepackage{tabularx}
\usepackage{wallpaper}
\usepackage{listings}
\usepackage{amsmath}

\usepackage{xcolor, colortbl}
\definecolor{ugentblue}{HTML}{164A7C}
\definecolor{gray}{HTML}{AAAAAA}
\definecolor{lightgray}{HTML}{FAFAFA}
\definecolor{grayborder}{HTML}{CCCCCC}
\definecolor{commentgreen}{HTML}{009900}

% Tables
\def\arraystretch{1.35}
\renewcommand{\tabularxcolumn}[1]{>{\small}m{#1}}
\newcommand{\hcell}[1]{
	\cellcolor{ugentblue}\color{white}\textbf{#1}
}

\makeatletter
\renewcommand\thesubsection{\@arabic\c@section.\@arabic\c@subsection}
\makeatother{}

\usepackage[hypertexnames=false]{hyperref}
\usepackage[numbered, depth=3]{bookmark}

\renewcommand{\headrulewidth}{0pt}
\pagestyle{fancy}
\fancyhf{}

\interfootnotelinepenalty=0

% Listings
\lstset{
	backgroundcolor=\color{lightgray},
	basicstyle=\footnotesize,
	commentstyle=\color{commentgreen},
	frame=single,
	keywordstyle=\color{blue},
	language=Java,
	numbers=left,
	numbersep=5pt,
	numberstyle=\tiny\color{gray},
	rulecolor=\color{black},
	stepnumber=1,
	stringstyle=\color{ugentblue},
	showspaces=false,
	showstringspaces=false
}

\begin{document}
	\begin{titlepage}
		%% Footer
		\thispagestyle{fancy}
		\fancyhf{}

		%% Page
		\hfill
		\begin{minipage}[t][0.9\textheight]{0.8\textwidth}
			\noindent
			\includegraphics[width=55px]{ugent-blue.png} \\[-1em]
			\color{ugentblue}
			\makebox[0pt][l]{\rule{1.3\textwidth}{1pt}}
			\par
			\noindent
			\textbf{\textsf{Vage Databanken}} \textcolor{gray}{\textsf{Academiejaar 2014-2015}}
			\vfill
				\noindent
			{\huge \textsf{Project Vage Databanken}}
			\vskip\baselineskip
			\noindent
			\textsf{\textbf{Jasper D'haene} \\
				\textbf{Florian Dejonckheere}}
		\end{minipage}
	\end{titlepage}

	%%% PAGE STYLE %%%
	\nopagecolor
	\renewcommand{\footrulewidth}{0.4pt}
	\headheight 45pt
	\ULCornerWallPaper{1}{header.png}
	\fancyfoot[C]{\thepage}

	%%% DOCUMENT %%%
	\section{Generiek vaagregelsysteem}
		\noindent Beschrijving van het \texttt{FuzzySystem}. \\

		\noindent In het project werd gebruik gemaakt van een enkele extra library: de \href{https://commons.apache.org/proper/commons-math/}{Apache Commons Mathematics} library. Deze voorziet een stabiele numerieke implementatie van verscheidene integratiemethoden. Omdat de integratietijd belangrijker is dan de precisie, werd gekozen voor de simpelste implementatie: de \texttt{MidPointIntegrator}.

	\section{Controllers}
		\subsection{SafeController}
			\noindent Beschrijving van de \texttt{SafeController}.

		\subsection{SpeedController}
			\noindent Beschrijving van de \texttt{SpeedController}.

		\subsection{RallyController}
			\noindent Beschrijving van de \texttt{RallyController}.

	\section{Performantie}
		\subsection{Keuze t-norm en t-conorm}
			%% Is de keuze van t-norm en t-conorm belangrijk voor de prestatie (i.e. de gemiddelde snelheid) van de wagen?

		\subsection{Robuustheid regelsysteem}
			%% Hoe robuust is uw regelsysteem voor verschillende races? Is de gemiddelde snelheid ongeveer dezelfde als u de wagen verschillende keren laat racen op een welbepaald parcours?

		\subsection{Minder performante parcours}
			%% Zijn er parcours waar uw controller minder goed presteert? Hoe komt dit?

		\subsection{Veiligheid}
			%% Hoe veilig is uw wagen, i.e. hoe dikwijls crasht uw wagen?

		\subsection{Exacte controller}
			%% Wat zijn de verschillen met een exacte controller (beter/slechter)?
\end{document}
